\documentclass[a4paper, 11pt]{article}
\usepackage[polish]{babel}
\usepackage[utf8]{inputenc}
\usepackage[T1]{fontenc}
\usepackage{listings}
\usepackage{enumerate}
\usepackage{times}

\title{Laboratoria 7}

\author{Krzysztof Żywiecki}

\begin{document}
	
	\maketitle
	
	\section*{Wstęp}
	
	Podpunkt pierwszy nie został zrobiony
	
	Drugi podpunkt wymaga ustawienia odpowiedniej wielkości bufora do zapisu i odczytu z kolejki. Na moim systemie domyślnie ta wielkość wynosi 8192 bajty, czyli po uwzględnieniu zmiennej PID, wiadomość powinna zajmować 8188 bajtów.
	
	\section*{Ćwiczenie 1}
	
	Zadanie polega na przerobieniu programu eliza.c tak, żeby działał w trybie klient serwer. Wynikowy program został stworzony dzięki przerobieniu programu server tak, żeby nie czekał na input z terminala, a zamiast tego pobierał informacje z funkcji \emph{respond}. Efekt uzyskano dzięki stworzeniu globalnej zmiennej \emph{char buffer[MESSAGE\_BUFFER\_SIZE]} i zastąpieniu wszystkich instrukcji \emph{printf} instrukcjami \emph{strcat}. 
	
	\section*{Ćwiczenie 2}
	
	Nie zrobione 
	
	
	
\end{document}