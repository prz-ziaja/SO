\documentclass[a4paper, 12pt]{article}
\usepackage[polish]{babel}
\usepackage[utf8]{inputenc}
\usepackage[T1]{fontenc}
\usepackage{listings}
\usepackage{enumerate}
\usepackage{times}
\usepackage{anysize}

\title{Laboratoria 6}
\author{Krzysztof Żywiecki}

\begin{document}

\maketitle

\section*{Ćwiczenie 1}

W ćwiczeniu pokazany jest przykład zastosowania funkcji pipe do ustalenia komunikacji między procesami. W przykładowym programie następuje zapis, a następnie odczyt danych z pośrednika. Po użyciu funkcji fork trzeba zamknąć w procesach odpowiednie deskryptory. Strona zapisująca oszczędza w ten sposób alokacje. Strona odczytująca z kolei musi zamknąć swój pipe do zapisu, gdyż przeciwna strona musi po skończeniu zapisu zasygnalizować EOF, co nie jest możliwe jak istnieje kilka zapisujących stron.

Funkcja \emph{fpathconf} dostarcza nam wielu informacji na temat ograniczeń systemu. Wielkość bufora określa ile bajtów może zostać wpisanych do \emph{pipe}

\section*{Ćwiczenie 2}
Program demonstracyjny w bardzo sprytny sposób łączy \emph{pipe[0]} z wejściem standardowym do procesu. Funkcja \emph{dup} tworzy kopię deskryptora na najniższy możliwy indeks. Po wykonaniu \emph{close(0)} jest to właśnie standardowe wejście. W programie dodano zamykanie deskryptorów: 0 w wątku rodzica, oraz 1 w wątku potomnym jak w poprzednim zadaniu. Dodatkowo po wykonaniu \emph{dup} można zwolnić w wątku potomnym deskryptor 0.

\section*{Ćwiczenie 3}
//Dokończone w piątek

W zadaniu mamy do wykonania 3 programy.

Pierwszy wymaga do stworzenia dwóch procesów potomnych wraz z metodami komunikacji między nimi. Proces główny wczytuje całość pliku i przekazuje go do potomnych procesów. Te odczytują tekst linijka po linijce, i liczą odpowiednio linijki.

Drugi program tworzy dwa procesy potomne. Pierwszy z procesów wykonuje proces seq z odpowiednimi argumentami. Drugi z nich odczytuje kolejne liczby i wysyła je przemnożone przez 5 do procesu głównego .

Trzeci program wykonuje po kolei 4 procesy: wczytujący nazwy użytkowników, obcinający je do nazwy użytkownika, sortujący je alfabetycznie i biorący tylko unikatowe wartości.


\end{document}