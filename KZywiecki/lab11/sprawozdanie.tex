\documentclass[a4paper, 11pt]{article}
\usepackage[polish]{babel}
\usepackage[utf8]{inputenc}
\usepackage[T1]{fontenc}
\usepackage{graphicx}
\usepackage{caption}
\usepackage{subcaption}
\usepackage{anysize}
\usepackage{listings}

\title{Systemy Operacyjne\\
	\Large{Laboratoria 11}}
\author{Krzysztof Żywiecki}

\begin{document}
	
	\maketitle
	
	\section*{Komunikacja bezpołączeniowa}
	Komunikację bezpołączeniową nawiązuje się za pomocą protokołu UDP. Stosuje się ją w wypadkach gdy ważna jest szybkość przesyłu danych, ale niekoniecznie przejmujemy się tym czy przesłane informacje będą kompletne. Jednym z przykładów zastosowania takiej komunikacji jest streaming filmów, gdzie zgubienie jednej klatki nie jest dużym problemem.
	
	Mimo braku kontroli zapewnianej przez protokół, aplikacja może implementować mechanizmy do sprawdzania czy przesłane dane są kompletne. Protokół TFTP robi to tak, że zarówno klient jak i serwer komunikują się na temat przesyłu danych: po połączeniu do serwera i potwierdzeniu połączenia, klient wysyła numerowane pakiety danych, a serwer odpowiada potwierdzeniem otrzymania każdego numerowanego pakietu. 
	
	\section*{Broadcasting}
	Z tego co zrozumiałem, aby wykonać broadcast trzeba wszystkie bity poza maską podsieci ustawić na 1, znaczy w sieci 200.200.200.0/24 adresem broadcast jest 200.200.200.255. Żeby zrobić broadcast do wszystkich użytkowników sieci LAN, możemy użyć adresu 255.255.255.255. Pamiętajmy że żeby zrobić broadcast, socket musi posiadać uprawnienia do broadcasta (SO\_BROADCAST)
	
\end{document}

