\documentclass[a4paper, 11pt]{article}
\usepackage[polish]{babel}
\usepackage[utf8]{inputenc}
\usepackage[T1]{fontenc}
\usepackage{graphicx}
\usepackage{caption}
\usepackage{subcaption}
\usepackage{anysize}
\usepackage{listings}

\title{Systemy Operacyjne\\
	\Large{Laboratoria 12}}
\author{Krzysztof Żywiecki}

\newcommand{\entry}[2]{\item\emph{#1} - #2}
\newcommand{\addrinfo}{\emph{addrinfo}}

\begin{document}

\maketitle

\section*{Hostent}

Hostent jest strukturą zawierającą informacje na temat hosta, oczywiście.
\begin{itemize}
	\entry{h\_name}{nazwa hosta}
	\entry{h\_aliases}{lista aliasów hosta}
	\entry{h\_addrtype}{typ adresu}
	\entry{h\_length}{długość adresu}
	\entry{h\_addr\_list}{lista adresów}
\end{itemize}
Przedstawione informacje brane są z pliku \emph{/etc/hosts}.

Program zadany w ćwiczeniu wypisuje hosty dostępne na urządzeniu z pomocą funkcji \emph{gethostent}. Wyjście tego programu jest podobne do zawartości pliku \emph{/etc/hosts}. Różnią się one tylko dwoma dodatkowymi wpisami w pliku, zapisanymi jako ipv6-allnodes, oraz ipv6-allrouters.

Sprawdzanie błędów wygląda podobnie jak w innych funkcjach. Jedyną różnicą jest to że zamiast zmiennej \emph{errno} ustawiane jest \emph{h\_errno}. Z tego powodu nie można bezpośrednio użyć funkcji \emph{perror}, zamiast tego trzeba zrobić blok switch w którym wypiszemy odpowiedni do błędu komunikat.

Żeby pobrać wszystkie adresy skorzystano z pola \emph{h\_addr\_list}, z którego po kolei zostały pobrane adresy. Dane były parsowane z użyciem już podanej pętli.

\subsection*{Getaddrinfo}

Funkcja \emph{getaddrinfo} zwraca liczbę całkowitą będącą komunikatem o błędzie, oraz strukturę \emph{addrinfo}. Funkcja przyjmuje nazwę hosta, nazwę serwisu, to znaczy numer portu, albo serwis typu ,,echo'', oraz strukturę ze wskazówkami \emph{addrinfo}.

W strukturze \addrinfo~pole \emph{ai\_protocol} może przyjąć wartości http, ftp, albo telnet.

Do sprawdzania błędów używamy zwracanej przez funkcję wartości i porównujemy ją z kodami błędów. Liczbę można też podać jako argument funkcji \emph{lwres\_gai\_strerror} żeby otrzymać wiadomość błędu, podobnie jak w funkcji \emph{strerror}

Flagi AI\_PASSIVE możemy użyć żeby utworzyć socket który czeka na żądanie połączenia.

Usługi oraz przypisane do nich porty znajdziemy w pliku \emph{/etc/services}

Przechodząc do programu, korzystanie ze struktury \addrinfo~jest nieco wygodniejsze niż w wypadku \emph{hostent}. Adresy IP są połączone za pomocą pola \emph{ai\_next}, tworząc listę. 

Żeby uzyskać nazwę kanoniczną hostów, trzeba odczytać pole \emph{ai\_canonname}. Żeby pole było ustawione, należy dodać flagę AI\_CANONNAME przy wywoływaniu funkcji \emph{getaddrinfo}.

\section*{Funkcje dotyczące sieci}

Funkcje set/get/endnetent przeszukują plik /etc/networks. 

Wyświetlanie nazw sieci jest bardzo podobne do wyświetlania wartości \emph{hostinfo}. Korzystamy z setnetent, a następnie odczytujemy po kolei nazwy ze struktur netent.

\section*{Usługi}

Serwisy, jak wszystkie poprzednie funkcje, korzystają z dedykowanego pliku /etc/services. Również odczytywanie serwisów jest niezwykle proste i ogranicza się do wczytywaniu w pętli zwracanych wartości funkcji \emph{getsetvant}.

\section*{Protokoły}
JW
\section*{Interfejsy}
Wyszukiwanie interfejsów jest równie proste, ale zamiast wczytywać struktury sekwencyjnie, wczytujemy od razu tablicę danych.

\end{document}